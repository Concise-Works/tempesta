\documentclass{article}
\input{Configs/configs}

\begin{document}

\noindent
\textcolor{black!50}{\texttt{CS235: pset-1, Section 1 \#2, 7, 11, 14 (Due: September 12th)}}

\vspace{2em}
\noindent
\underline{\textbf{Exercise 1.2.}} Let \( n \) be a composite integer.
Show that there exists a prime \( p \) dividing \( n \), with \( p \leq n^{1/2} \).

\begin{greenbox}
    \textit{Proof:} Let $n$ be a composite integer $\therefore$ $n = ab$ for some integers $a, b\mid 1 < a, b < n$.\\
    $a,b \leq n^{1/2}$ must hold, or else $ab > n$.\\

    \noindent
    $n>0$ $\therefore n$ can factor to powers of primes $p$ (Fundamental Theorem of Arithmetic).
    If $p$ is a composite, factor again until a single prime $p'$ is found. $p'\mid p$ then $p' \mid n$ and $p' \leq n^{1/2}$. \hfill \(\blacksquare\)
\end{greenbox}

\noindent
\textbf{Theorem 1.5.} Let \( a, b \in \mathbb{Z} \) with \( b > 0 \), and let \( x \in \mathbb{R} \). Then there exist unique \( q, r \in \mathbb{Z} \) such that \( a = bq + r \) and \( r \in [x, x + b) \).\\

    \noindent
    \underline{\textbf{Exercise 1.7.} }Show that Theorem 1.5 also holds for the interval \( (x, x + b] \). Does it hold in general for the intervals \([x, x + b]\) or \( (x, x + b) \)?
\begin{greenbox}
    \textit{Proof:} Thm. 1.5 is the division algorithm, for \( a, b \in \mathbb{Z} \) with \( b > 0 \), and let \( x \in \mathbb{R} \). Then there exist unique \( q, r \in \mathbb{Z} \) such that \( a = bq + r \) and \( r \in [x, x + b) \).\\

    \noindent
    $r$, the remainder demonstrates the interval \([x, x + b)\).
            This interval represents the residue class of $a$ modulo $b$ with shifts in mind. This class is of length $b$.
            The question of whether the interval \((x, x + b]\) holds is if it sustains the same length of $b$ to represent the residue classes, which it does.\\

    \noindent
    The intervals \([x, x + b]\) and \((x, x + b)\) do not hold such lengths. \hfill \(\blacksquare\)
\end{greenbox}

\noindent
\underline{\textbf{Exercise 1.11.}} Let \( n \) be an integer. Show that if \( a, b \) are relatively prime integers, each of which divides \( n \), then \( ab \) divides \( n \).

\begin{greenbox}
    \textit{Proof:} Let $a,b$ be relatively prime integers, then there exists $s,t$ such that $as + bt = 1$.
    If $a\mid n$ then $n = ak$ for some integer $k$. If $b\mid n$ then $n = bq$ for some integer $q$.
    $ n = ak = bq$, multiplying both equations $(ak)(bq) = (ak)(bq) = (ab)(kq)$, $kq$ is some integer say $j$.
    Therefore $ab(j) = n$ and $ab\mid n$. \hfill \(\blacksquare\)
\end{greenbox}

\newpage
\noindent
\underline{\textbf{Exercise 1.14.}} Let \( p \) be a prime and \( k \) an integer, with \( 0 < k < p \). Show that the binomial coefficient
\[
    \binom{p}{k} = \frac{p!}{k!(p - k)!},
\]
which is an integer (see §A2), is divisible by \( p \).

\begin{greenbox}
    \textit{Proof:} The binomial coefficient \(\binom{p}{k} = \frac{p!}{k!(p - k)!}\) is an integer, which when expanded:\\
    \[
        \frac{p(p-1)...(p-k+1)}{k!}
    \]
    $(p-1)...(p-k+1)$ is some integer $m$. So $\frac{pm}{k!}$ then $p\mid pm$, $p$ divides the numerator.
    Since $k$ is less than $p$, $p$ does not divide $k!$. Therefore $k$ won't cancel the $p$ in the numerator.
    Since the result is an integer, a factor of $p$ must still be present. \hfill \(\blacksquare\)


\end{greenbox}



\end{document}
